\chapter{引言}
\label{cha:introduction}
有许多种方式来描述距离几何问题, 我们采用图的语言. 

\begin{Prob}[等式约束的距离几何问题]
  对于图$G=(V,E)$, 其中$V$是顶点几何, $E$是边的集合, 给定每一条边 $(i,j)\in E$ 的长度为 $d_{ij}$, 求解$d$ 维欧式空间中的点的坐标 $\xn$,使得
\begin{equation}
  \|x_i-x_j\|=d_{ij}, \quad (i,j)\in E.
\end{equation}
\end{Prob}

在理想情况下, 给定的距离是没有误差(无噪音)的, 
此时问题叫做精确距离的距离几何问题, 否则称为不精确(带噪音)的距离几何问题.
一个更加实际的情形是, 给定的不是每一条边上的距离 $d_{ij}$, 
而是该距离的上界估计 $u_{ij}$ 和下界估计  $l_{ij}$.
此时问题变为求 $x_i\in \Real ^d$, 满足
\begin{equation}
  l_{ij}\leq \|x_i-x_j\| \leq u_{ij}, \quad (i,j)\in E.
\end{equation}
此时问题叫做给定上下界的距离几何问题.
在本文中, 我们会涉及到上下界的问题, 但重点研究带噪音的距离几何问题.


\section{距离几何问题的应用}
\label{sec:application}
距离几何问题在多个领域有着广泛的应用, 
本文仅列出其中最重要的几个例子, 实际应用包括但不限于这些.


\subsection{画图}
在画图 (Graph Drawing) 领域, 通常叫做图的实现(Graph Realization)问题~\cite{Gansner2005}.
在这个问题中, 我们的任务是画图使得点之间的距离满足事先给定的权重. 
这个问题跟我们后面提到的几个应用的区别是, 这些理论上的图
有时候会有多解的存在, 为了使画出来的图满足美观性的要求通常会引入其他约束条件.

\subsection{蛋白质折叠} 
这个术语翻译自 Protein Folding, 
在其他文献中也有被叫做蛋白质结构确定 (Protein Structure Determination)~\cite{Braun1987,Sit2011,Voller2013}, 
或是分子构象问题 (Molecular Conformation Problem)~\cite{Crippen1988,Biswas2008,Fang2013}, 
或是分子距离几何问题 (Molecular Distance Geometry Problem)~\cite{Dong2002,Dong2003,Carvalho2008}.

在这个问题中, 给定原子之间的部分距离, 我们需要给出蛋白质的三维结构.
在实际应用中, 这些距离通常是由实验测得的, 比如核磁共振(Nuclear Magnetic Resonace)
或X射线结晶技术(X-ray Crystallography), 或者通过一些生物学信息估算, 比如键长和键角.
在目前的应用中, 核磁共振是获得距离的主要技术.
由于蛋白质的很多重要性质都与三维结构紧密相关, 
所以这个问题非常具有实际意义.
    
\subsection{传感器网络定位} 
另一个非常重要的应用是传感器网络定位~\cite{Akyildiz2002,Chong2003,Mao2007,Yick2008}.
传感器非常便宜和方便, 它是用来环境监测, 动物管理等活动的一种有效的工具, 
甚至应用在军事行动中, 用于远程探测地面信息.
在这些应用中, 通常每一个个体都装备有传感器, 它可以用来
发射信号, 收集和简单地处理信息.
基于传感器的很多应用都是位置相关的, 它的第一步就是确定各个体的地理位置.
传感器能够在一定的射程内发射和接收信号, 这样附近的传感器就可以通过到达时间差
或信号的衰减来测量距离, 通常情况下后者更精确, 
对于前者来说时间同步是一大影响精度的因素.

这个应用更其他几个的一个重要差别是, 在此应用中, 小部分传感器结点的精确位置
是提前知道的, 这部分结点也被成为锚节点 (anchors).
在很多其他应用中, 结点的相对位置, 也就是整体结构,
但借助于这些锚节点, 传感器定位中可以得到个体的绝对位置.
 
\subsection{其他应用}
距离几何问题还有其他许多应用, 比如地下巷道定位.
在煤炭开采等地下巷道中, 由于卫星信号的缺失, GPS 这种常规定位手段就会失效.
如果让工人都携带传感器, 并在巷道中辅以锚节点, 
我们就能确定所有工人的位置, 这对日常管理和事故中的紧急救援都是非常有用的.

现今移动电子设备如手机等增长迅速, 我们可以大胆设想, 
在未来, 我们能够通过这些设备的近场通讯, 测量距离, 从而实现室内精确定位.
在一些大型商场中, 已经有些商家做到了这一点, 但主要是基于移动设备
跟锚节点的通讯来实现的. 未来如果移动设备也互相通信,
组成一个大的移动网络, 将有助于更快更精确地定位.

\subsection{一些注记}
在蛋白质折叠和传感器网络定位应用中, 由于核磁共振技术和无线信号强度
的限制, 我们能够测得的距离通常都是局部的.
这样, 从总体上来看, 我们已知的距离信息就是非常稀疏的,
这里稀疏的意义是, 已知的距离是所有的成对 (pairwise) 距离中的很小一部分.
这也是问题的一个难点所在, 关于这点我们会在后文的复杂性分析和数值实验中再深入讨论.

上面提到了不少应用, 这也是本文作者选择这个课题作为博士研究方向的出发点之一.
我们希望提出的算法不光要有理论上的意义, 也能真正用来解决实际问题.
尽管实际的应用问题要比理论研究复杂, 还需要考虑成本等实际因素,
但算法还是其中非常重要的一部分, 这正是我们努力的方向.

通过上面的讨论, 我们可以看到, 
尽管都叫做``距离几何问题'', 其具体形式可能在以下几个方面存在差异:
\begin{itemize}
  \item 是否存在锚节点~?
  \item 等式约束还是上下界约束~?
  \item 已知全部距离还是部分距离~?
\end{itemize}
问题形式的不同将导致截然不同的难度和求解算法.
另外强调一点, 我们假设需要求解的点所在空间的维数是确定的, 已知的.


 
\section{本文主要内容}

在本文中, 我们的目标是求解蛋白质折叠问题,
也就是, 不带锚节点, 等式约束, 
已知部分 (其实非常稀疏) 距离信息的三维空间的
距离几何问题.


