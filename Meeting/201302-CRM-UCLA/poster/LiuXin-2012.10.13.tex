\documentclass[mathserif,compress,orchid,cjk]{beamer}
\usepackage{txfonts}

\mode<presentation> {
  \usetheme{classic}%{Warsaw}{Madrid}{Madrid}
  \useinnertheme{umbcboxes}
   \setbeamercovered{transparent}
}

\usepackage{CJK}
\usepackage{hyperref}
\hypersetup{CJKbookmarks=true}

\usepackage{amsmath,amssymb,amsfonts}
\usepackage{color,xcolor}
\usepackage{graphicx}
\usepackage{manfnt}
\usepackage{pgf}
\usepackage{booktabs}
\DeclareMathOperator*{\argmin}{arg\,min}
\newcommand{\song}{\CJKfamily{song}}    % ����   (Windows�Դ�simsun.ttf)
%%%%%%%%%%%%%%%%%%%%%%%%%%%%%%%%%%%%%%%%%%%%%%%%%%%%%%%%%%%%%%%%%%%%%%%%%%%%%%%%%%%%%%%%%%%%%%%%%%%%%%%%%
%%%%%%%%%%%%%%%%%%%%%%%%%%%%%%%%%%%%%%%%%%%%%%%%%%%%%%%%%%%%%%%%%%%%%%%%%%%%%%%%%%%%%%%%%%%%%%%%%%%%%%%%%

\begin{document}
\definecolor{blue}{rgb}{0,0,.65}
\definecolor{myblue}{rgb}{0,0,.5}
\definecolor{mygreen}{rgb}{0,.5,0}
\definecolor{red}{rgb}{1,0,0}
\definecolor{myred}{rgb}{.5,0,0}

\newtheorem{dingyi}{Definition~}[section]
\newtheorem{dingli}{Theorem~}[section]
\newtheorem{yinli}{Lemma~}[section]
\newtheorem{tuilun}{Corollary~}[section]
\newtheorem{mingti}{Proposition~}[section]
\newtheorem{caixiang}{Hypothesis~}[section]
\newtheorem{jiashe}{Assumption~}[section]
\newtheorem{lizi}{Instance~}[section]
%\beamertemplateshadingbackground{blue!5}{white!15}
\renewcommand{\proof}{\noindent {\bf Proof.} \ \ }
\renewcommand{\endproof}{{\bf Q.E.D.}}
\newcommand{\zz}{^{\text{T}}}
\newcommand{\fzz}{^{-\text{T}}}
\newcommand{\ff}{_{\text{F}}}
\newcommand{\fs}{^2_{\text{F}}}
\newcommand{\eproof}{$\quad \Box$}
\newcommand{\mR}{\mathbb{R}}
\newcommand{\R}{\mathbb{R}}
\newcommand{\st}{\;\mathrm{s.t.}\;}
\newcommand{\bT}{\mathbb{T}}
\newcommand{\Lone}{\Lambda_1}
\newcommand{\Ltwo}{\Lambda_2}
\newcommand{\LA}{\mathcal{L}_{(\beta_1,\ \beta_2)}}
\newcommand{\Po}{\mathcal{P}_\Omega}
\newcommand{\Pob}{\mathcal{P}_{\Omega^c}}
\newcommand{\Zo}{Z_{\omega}}
\newcommand{\La}{\mathcal{L}_{\alpha}}
\newcommand{\bw}{\mathbf{w}}
\newcommand{\bh}{\mathbf{h}}
\newcommand{\Lb}{\mathcal{L}_{\beta}}


%===================================================%
\makeatletter
\renewcommand\theequation{\thesection.\arabic{equation}}
\@addtoreset{equation}{section} \makeatother

\title[ADMM: A Powerful Tool]
{Alternating direction method of multiplier:\\ a powerful tool for difficult optimization problems}

\author[Xin~Liu]{\bf Xin Liu}

%\institute{\xiaosihao\hei\textcolor[rgb]{0.85,0.42,0.00}{LSEC ICMSEC AMSS CAS}}
\institute[AMSS]{State Key Laboratory of Scientific and Engineering Computing\\
Institute of Computational Mathematics and Scientific/Engineering Computing\\
Academy of Mathematics and Systems Science\\
Chinese Academy of Sciences, China}%\\[1.2em]

\date[October 13, 2012]{\scriptsize \rm Fifteenth Chinese-American
Kavli Frontiers of Science Symposium\\[0.6em]
\footnotesize {\tt Arnold and Mabel Beckman Center, Irvine, CA}\\[0.6em] October 13, 2012}

\setbeamertemplate{background}{
\pgfputat{\pgfxy(11.85,-0.96)}{\pgfbox[left,base]{\pgfimage[height=1.01cm,width=1cm]{tinylogo}}}}


%%%%%%%%%%%%%%%%%%%%%%%%%%%%%%%%%%%%%%%%%%%%%%%%%%%%%%%%%%%%%%%%%%%%%%%%%%%%%%%%%%%%%%%%%%%%%%1.1
\begin{frame}
  \titlepage
\end{frame}

%%%%%%%%%%%%%%%%%%%%%%%%%%%%%%%%%%%%%%%%%%%%%%%%%%%%%%%%%%%%%%%%%%%%%%%%%%%%%%%%%%%%%%%%%%%%%%1.1
\begin{frame}
\frametitle{Alternating direction method of multiplier:\\ a powerful tool for difficult optimization problems}
\vspace{-5mm}
\begin{center}\footnotesize
Xiu Liu (liuxin@lsec.cc.ac.cn)\\
{\rm\scriptsize State Key Laboratory of Scientific and Engineering Computing\\
Academy of Mathematics and Systems Science\\
Chinese Academy of Sciences, China}\\[2mm]
{\scriptsize Fifteenth Chinese-American Kavli Frontiers of Science Symposium\\
Irvine, California, U.S. October 13, 2012}
\end{center}  
\vspace{-3mm}
\begin{minipage}{40mm}
\begin{displaybox}{40mm}
``{\color{mygreen}Divide}" and ``{\color{myred}Conquer}" \\
\indent \quad\, $\Updownarrow$ \hspace{1.78cm}   $\Updownarrow$\\
\indent\,\,``{\color{mygreen}Split}" \, and ``{\color{myred}Alternate}"
\end{displaybox}
\end{minipage}
$\Rrightarrow$
\begin{minipage}{60mm}
\begin{displaybox}{60mm}\footnotesize
{\color{red}Difficult optimization models}
\begin{itemize}
\item nonlinearity, nonconvexity
\item nondifferentiable term
\item combinatorial objective or constraints 
\end{itemize}
\end{displaybox}
\end{minipage}
\begin{minipage}{60mm}
\begin{displaybox}{60mm}
\begin{itemize}\scriptsize
\item {\color{mygreen}Splitting} brings {\color{myblue} easy subproblems};
\item {\color{blue} Augmented Lagrangian function} penalizes the {\color{myblue} equality
constraints}
\item {\color{myred}Alternating} solves the {\color{myblue}split targets} in turn
\end{itemize}
\end{displaybox}
\end{minipage}
$\circlearrowleft$
\begin{minipage}{40mm}
\begin{displaybox}{40mm}\footnotesize
{\color{red} Three application instances}
\begin{itemize}
\item phase retrieval
\item portfolio optimization
\item matrix
factorization
\end{itemize}
\end{displaybox}
\end{minipage}
\end{frame}

\end{document}
