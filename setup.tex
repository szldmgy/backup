% File Name: SetUp.tex
% Function: Make main settings of the document.


%%%%%%%%%%%%%%%%%%%%%%%%%%%%%% BEGIN-Packages %%%%%%%%%%%%%%%%%%%%%%%%%%%%%
\usepackage{amsmath,amsthm,amssymb,amsfonts}
\usepackage{mathrsfs}
\usepackage{graphicx}
\usepackage{enumerate}
\usepackage{booktabs}
\usepackage{tabularx}
\usepackage{multirow,multicol}
\usepackage{eqlist}
\usepackage{url}
\usepackage{cases}
\usepackage{subfigure} 
\usepackage{boxedminipage}
%\usepackage[section]{placeins}
\usepackage{placeins}
\usepackage{float}
\usepackage{longtable}
\usepackage{rotating}
\usepackage{verbatim}
\usepackage[ruled,lined,linesnumbered]{algorithm2e}
\usepackage{afterpage}  %在table和figure环境的标题中加脚注, footnotemark
\usepackage{changepage} %配合adjustwidth环境整体缩进
%\usepackage{titlesec}

%%%%%%%%%%%%%%%%%%%%%%%%%%% listing package %%%%%%%%%%%%%%%%%%%%%%%%%%%%%%%%%%%
\usepackage{listings}
\lstset{ %
extendedchars=false,            % Shutdown no-ASCII compatible
language=Matlab,                % choose the language of the code
basicstyle=\footnotesize\tt,    % the size of the fonts that are used for the code
tabsize=3,                            % sets default tabsize to 3 spaces
%numbers=left,                   % where to put the line-numbers
numberstyle=\tiny,              % the size of the fonts that are used for the line-numbers
stepnumber=1,                   % the step between two line-numbers. If it's 1 each line
                                % will be numbered
numbersep=5pt,                  % how far the line-numbers are from the code   %
keywordstyle=\color[rgb]{0,0,1},                % keywords
commentstyle=\color[rgb]{0.133,0.545,0.133},    % comments
stringstyle=\color[rgb]{0.627,0.126,0.941},      % strings
backgroundcolor=\color{white}, % choose the background color. You must add \usepackage{color}
showspaces=false,               % show spaces adding particular underscores
showstringspaces=false,         % underline spaces within strings
showtabs=false,                 % show tabs within strings adding particular underscores
frame=single,                   % adds a frame around the code
captionpos=b,                   % sets the caption-position to bottom
breaklines=true,                % sets automatic line breaking
breakatwhitespace=false,        % sets if automatic breaks should only happen at whitespace
title=\lstname,                 % show the filename of files included with \lstinputlisting;
                                % also try caption instead of title
mathescape=true,escapechar=?    % escape to latex with ?..?
escapeinside={\%*}{*)},         % if you want to add a comment within your code
%columns=fixed,                  % nice spacing
%morestring=[m]',                % strings
%morekeywords={%,...},%          % if you want to add more keywords to the set
%    break,case,catch,continue,elseif,else,end,for,function,global,%
%    if,otherwise,persistent,return,switch,try,while,...},%
}
%%%%%%%%%%%%%%%%%%%%%%%%%%%%%%%% END-Packages %%%%%%%%%%%%%%%%%%%%%%%%%%%%%


%%%%%%%%%%%%%%%%%%%% BEGIN-Theorem-like Environments %%%%%%%%%%%%%%%%%%%%%
\theoremstyle{plain}
%\newtheorem{Thm}{定理}[section]
\newtheorem{Thm}{\indent 定理}[chapter]
\newtheorem{Prop}[Thm]{\indent 命题}
\newtheorem{Lem}[Thm]{\indent 引理}
\newtheorem{Coro}[Thm]{\indent 推论}
\newtheorem{Rem}[Thm]{\indent 注}
\newtheorem{Asm}[Thm]{\indent 假设}
\newtheorem{Alg}[Thm]{\indent 算法}
\newtheorem{Cond}[Thm]{\indent 条件}
\newtheorem{Prob}[Thm]{\indent 问题}
\newtheorem{Def}[Thm]{\indent 定义}
\renewcommand{\proofname}{\indent {\heiti 证明}}
%\numberwithin{equation}{section}
\numberwithin{equation}{chapter}
%%%%%%%%%%%%%%%%%%%%% END-Theorem-like Environments %%%%%%%%%%%%%%%%%%%%%%

\graphicspath{{figures/}} % Set the directory where figures are saved.



%%%%%%%%%%%%%%%%%%%%%%%%%% BEGIN-New Commands %%%%%%%%%%%%%%%%%%%%%%%%%%%%
\newcommand{\chap}[1]{\textbf{Chapter} #1}

\newcommand{\ps}{\ \,}
\newcommand{\vect}{\textnormal{vec}}

\newcommand{\Gmf}{\Gamma\hskip-1mm}

\newcommand{\topcaption}{%
\setlength{\abovecaptionskip}{0pt}%
\setlength{\belowcaptionskip}{10pt}%
\caption}
\newcommand{\Bcaptionskip}{10pt}

\newcommand{\subfigwidth}{0.485\textwidth}
\newcommand{\namewidth}{1in}
\newcommand{\ski}{0.5em}
\newcommand{\comp}{\circ}
\newcommand{\hdp}{\!\comp\!}
\newcommand{\hdsq}[1]{#1^{\comp2}}
\newcommand{\hdsqrt}[1]{#1^{\comp\frac{1}{2}}}
\newcommand{\comps}{\!\comp\!}
\newcommand{\nxn}[0]{{n\times n}}
\newcommand{\dist}{\mathrm{dist}}
\newcommand{\diag}{\mathrm{diag}}
\newcommand{\pinv}[1]{#1^+}
%\newcommand{\pinv}[1]{#1^\dagger}
\newcommand{\tpinv}[1]{#1^{\dagger\textnormal{T}}}
\newcommand{\hpinv}[1]{#1^{\frac{1}{2}\dagger}}
\newcommand{\iti}[1]{\textnormal{(#1.)}}

\renewcommand{\baselinestretch}{1.3}
\renewcommand{\qedsymbol}{$\blacksquare$}
\renewcommand{\theenumi}{\textnormal{\alph{enumi}.)}}
\renewcommand{\labelenumi}{\theenumi}

\DeclareMathAlphabet{\mathcal}{OMS}{cmsy}{m}{n}    % Use standard calligrafic font of mathcal, instead of the rsfs one.
\newcommand{\fsp}[1]{\mathcal{#1}}       % Use mathcal font for function spaces.
\newcommand{\Qfs}{\fsp{Q}}      
\newcommand{\solver}{\mathcal{S}}
\newcommand{\prob}{\mathcal{P}}
%%%%%%%%%%%%%%%%%%%%%%%%%%%%%%%%%
\newcommand{\funl}[1]{\mathscr{#1}}      % Use Euler mathscr font for functionals.
\newcommand{\Fun}{\funl{F}}     
%%%%%%%%%%%%%%%%%%%%%%%%%%%%%%%%%
\DeclareMathAlphabet{\mathpzc}{OT1}{pzc}{m}{it}   
%\newcommand{\setrn}[1]{\mathpzc{#1}}     % Use mathpzc font for subsets of R^n.
\newcommand{\setrn}[1]{\mathcal{#1}}     % Use mathpzc font for subsets of R^n.
\newcommand{\set}{\setrn}
\newcommand{\Space}{\setrn}
\newcommand{\ball}{\setrn{B}}
\newcommand{\IntS}{\setrn{I}}
%%%%%%%%%%%%%%%%%%%%%%%%%%%%%%%%%
\newcommand{\kaibox}[1]{\mbox{\kaishu #1}}
\newcommand{\alg}[1]{\mbox{\textnormal{\texttt{#1}\ }}}      % Use \texttt for algrithm/software names.
\newcommand{\dfo}{\alg{DFO}}
\newcommand{\newuoa}{\alg{NEWUOA}}
\newcommand{\lfn}{\alg{LFN}}
\newcommand{\newuoam}{\alg{NEWUOAm}}
\newcommand{\newuoas}{\alg{NEWUOAs}}
\newcommand{\uobyqa}{\alg{UOBYQA}}
\newcommand{\bobyqa}{\alg{BOBYQA}}
\newcommand{\mnh}{\alg{MNH}}
\newcommand{\mads}{\alg{MADS}}
\newcommand{\gps}{\alg{GPS}}
\newcommand{\apps}{\alg{APPSPACK}}
\newcommand{\appspack}{\alg{APPSPACK}}
\newcommand{\nmsmax}{\alg{NMSMAX}}
\newcommand{\imfil}{\alg{IMFIL}}
\newcommand{\condor}{\alg{CONDOR}}
\newcommand{\cobyla}{\alg{COBYLA}}
\newcommand{\orbit}{\alg{ORBIT}}
\newcommand{\dfls}{\alg{DFLS}}
\newcommand{\boosters}{\alg{BOOSTERS}}
\newcommand{\SYMB}{\alg{SYMB}}
\newcommand{\ESYMBS}{\alg{ESYMBS}}
\newcommand{\ESYMBP}{\alg{ESYMBP}}
\newcommand{\fmins}{\alg{fminsearch}}
\newcommand{\lancelot}{\alg{LANCELOT}}
\newcommand{\subi}{\alg{SUBSPACE1}}
\newcommand{\subii}{\alg{SUBSPACE2}}
\newcommand{\subiii}{\alg{SUBSPACE3}}
\newcommand{\subspace}{\alg{SUBSPACE}}
\newcommand{\psub}{\alg{PSUB}}
\newcommand{\cuter}{\alg{CUTEr}}
\newcommand{\matlab}{\textsc{Matlab}\ }  % Use \textsc for Matlab.
\newcommand{\subrout}[2]{\textnormal{\texttt{#1}}\textnormal{(}#2\textnormal{)}}
%%%%%%%%%%%%%%%%%%%%%%%%%%%%%%%%%
\newcommand{\Real}{\mathbb{R}}
\newcommand{\Complex}{\mathbb{C}}
\renewcommand{\Re}{\mathrm{Re}}
\renewcommand{\Im}{\mathrm{Im}}
\newcommand{\Tran}[1]{#1^\mathrm{T}}
\newcommand{\tF}{\textnormal{F}}
\newcommand{\st}{\textnormal{s.t.}}
\newcommand{\Tr}{\mathrm{Tr}}
\newcommand{\Span}{\mathrm{span}}
\newcommand{\sign}{\mathrm{sign}}
\newcommand{\ran}{\mathcal{R}}
%\newcommand{\Sob}[2]{W_#1^#2}
\newcommand{\Sob}[2]{{H^#2}}
\newcommand{\SobH}[2]{{H^{#2,#1}}}
%%%%%%%%%%%%%%%%%%%%%%%%%%%%%%%%%%
\newcommand{\md}{\,\mathrm{d}}
\newcommand{\me}{\mathrm{e}}
\newcommand{\betaf}{\beta\!}
\newcommand{\Gammaf}{\Gamma\!}
\newcommand{\ellp}{\ell_p}
\newcommand{\ellt}{\ell_2}
\newcommand{\ello}{\ell_1}
\newcommand{\elli}{\ell_\infty}
\newcommand{\radius}{r}
\newcommand{\mean}{\mathrm{mean}}
\newcommand{\std}{\mathrm{std}}
\newcommand{\rstd}{\mathrm{rstd}}


%%%%%%%%%%%%%%%%%%%%%%%%%%%%%%%%%%%%%%%%%%%
% szl thesis
\newcommand{\xn}{x_1,x_2,\ldots,x_n} 
\newcommand{\be}{\begin{equation}}
\newcommand{\ee}{\end{equation}}
\newcommand{\ba}{\begin{array}}
\newcommand{\ea}{\end{array}}
\newcommand{\bc}{\begin{center}}
\newcommand{\ec}{\end{center}}
\newcommand{\bl}{\begin{flushleft}}
\newcommand{\el}{\end{flushleft}}
%\newcommand{\pozhe}{\raisebox{0.5mm}{------}}
\newcommand{\pozhe}{---\hspace*{-2mm}---} %中文破折号
\newcommand{\hd}[1]{\multicolumn{1}{c}{#1}}
%%%%%%%%%%%%%%%%%%%%%%%%%%%%%%%%%%%%%%%%%%%


\def\mathllap{\mathpalette\mathllapinternal}
\def\mathllapinternal#1#2{%
\llap{$\mathsurround=0pt#1{#2}$}%
}
\def\clap#1{\hbox to 0pt{\hss#1\hss}}
\def\mathclap{\mathpalette\mathclapinternal}
\def\mathclapinternal#1#2{%
\clap{$\mathsurround=0pt#1{#2}$}%
}
\def\mathrlap{\mathpalette\mathrlapinternal}
\def\mathrlapinternal#1#2{%
\rlap{$\mathsurround=0pt#1{#2}$}%
}


%%%%%%%%%%%%%%%%%%%%%%%%%%%%END-NewCommands%%%%%%%%%%%%%%%%%%%%%%%%%%%%
